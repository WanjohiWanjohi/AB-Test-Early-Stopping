%%%%%%%%%%%%%%%%%%%%%%%%%%%%%%%%%%%%%%%%%
% Short Sectioned Assignment
% LaTeX Template
% Version 1.0 (5/5/12)
%
% This template has been downloaded from:
% http://www.LaTeXTemplates.com
%
% Original author:
% Frits Wenneker (http://www.howtotex.com)
%
% License:
% CC BY-NC-SA 3.0 (http://creativecommons.org/licenses/by-nc-sa/3.0/)
%
%%%%%%%%%%%%%%%%%%%%%%%%%%%%%%%%%%%%%%%%%

%----------------------------------------------------------------------------------------
%	PACKAGES AND OTHER DOCUMENT CONFIGURATIONS
%----------------------------------------------------------------------------------------

\documentclass[paper=a4, fontsize=11pt]{scrartcl} % A4 paper and 11pt font size

\usepackage[T1]{fontenc} % Use 8-bit encoding that has 256 glyphs
\usepackage{fourier} % Use the Adobe Utopia font for the document - comment this line to return to the LaTeX default
\usepackage[english]{babel} % English language/hyphenation
\usepackage{amsmath,amsfonts,amsthm} % Math packages

\usepackage{lipsum} % Used for inserting dummy 'Lorem ipsum' text into the template
\usepackage{bbding}
\usepackage{multirow}

\usepackage{sectsty} % Allows customizing section commands
\allsectionsfont{\centering \normalfont\scshape} % Make all sections centered, the default font and small caps

\usepackage{fancyhdr} % Custom headers and footers
\pagestyle{fancyplain} % Makes all pages in the document conform to the custom headers and footers
\fancyhead{} % No page header - if you want one, create it in the same way as the footers below
\fancyfoot[L]{} % Empty left footer
\fancyfoot[C]{} % Empty center footer
\fancyfoot[R]{\thepage} % Page numbering for right footer
\renewcommand{\headrulewidth}{0pt} % Remove header underlines
\renewcommand{\footrulewidth}{0pt} % Remove footer underlines
\setlength{\headheight}{13.6pt} % Customize the height of the header

\numberwithin{equation}{section} % Number equations within sections (i.e. 1.1, 1.2, 2.1, 2.2 instead of 1, 2, 3, 4)
\numberwithin{figure}{section} % Number figures within sections (i.e. 1.1, 1.2, 2.1, 2.2 instead of 1, 2, 3, 4)
\numberwithin{table}{section} % Number tables within sections (i.e. 1.1, 1.2, 2.1, 2.2 instead of 1, 2, 3, 4)

\setlength\parindent{0pt} % Removes all indentation from paragraphs - comment this line for an assignment with lots of text

\usepackage{amsmath}

%----------------------------------------------------------------------------------------
%	TITLE SECTION
%----------------------------------------------------------------------------------------

\newcommand{\horrule}[1]{\rule{\linewidth}{#1}} % Create horizontal rule command with 1 argument of height

\title{	
\normalfont \normalsize 
\textsc{} \\ [25pt] % Your university, school and/or department name(s)
\horrule{0.5pt} \\[0.4cm] % Thin top horizontal rule
\huge Evaluation of Early Stopping in A/B Testing  \\ % The assignment title
\horrule{2pt} \\[0.5cm] % Thick bottom horizontal rule
}

\author{Shan Huang} % Your name

\date{\normalsize\today} % Today's date or a custom date

\begin{document}

\maketitle % Print the title

%----------------------------------------------------------------------------------------
%	PROBLEM 1
%----------------------------------------------------------------------------------------

This documentation shows the evaluation of different early stopping algorithms. 

\section{Problem}
Given samples $\textbf{x}$ from treatment group, samples $\textbf{y}$ from control group, We are interested in whether there is a significant difference  between the mean the two variants $\delta = \mu(y)-\mu(x)$.

To save the cost of long-running experiments, we want to stop early if we are already certain that there is a significant result.

\section{Significance Analysis}
\label{sec:byt}
Given $H_0$ represents the null hypothesis of no difference, $H_1$ represents the alternative hypothesis meaning there is a difference, we can draw the conclusion of whether the result is statistically significant using either:

\begin{itemize}  
\item \textbf{Confidence interval}: If 0 is outside confidence interval of $\delta$, it is statistically significant. Vice versa.
\item or \textbf{credible interval}: Credible interval is the Bayesian version of confidence interval. If 0 is outside credible interval of $\delta$, it is statistically significant. Vice versa.
\item or \textbf{Bayes factor}: Bayes factors higher than 3 can be interpreted as support for the alternative hypothesis (significant difference), whereas values smaller than 1/3 can be interpreted as support for the null hypothesis (significant no difference). Values between 1/3 and 3 are inconclusive. 
\end{itemize}

The ability of each metric, i.e., types of significance it can detect, is shown in the following table.
\begin{center}
  \begin{tabular}{ | r | c | c | c | c | }
    \hline
    & paradigm & significant $H_1$ & significant $H_0$ & no significant result \\ \hline
    \textbf{confidence interval} & frequentist &  \Checkmark &  & \Checkmark\\ \hline
    \textbf{credible interval} & Bayesian & \Checkmark &  & \Checkmark\\ \hline
    \textbf{Bayes factor} & Bayesian & \Checkmark & \Checkmark & \Checkmark\\
    \hline
  \end{tabular}
\end{center}

For simplicity, we draw a binary conclusion that either there is a significant difference or not. In other words, the first column means that there is a significant difference, combining the second and third column means that there is no significant difference.

It is worth noting that a \textbf{typical conclusion of Bayes factor} would be "\emph{There is a significant difference corresponds to Cauchy prior and a threshold of Bayes factor=3}". This might be quite difficult to explain to non-tech users. While the \textbf{conclusion based on interval} can be more intuitive such as "\emph{You can be 95\% sure that the significant difference is not due to chance}".


\section{Early Stopping Criteria}
We can stop the experiment by either
\begin{itemize}  
\item \textbf{Confidence interval}: If 0 is outside confidence interval of $\delta$, stop. Calculate the significance level for each day based on group sequential method.
\item or \textbf{credible interval}: If 0 is outside credible interval of $\delta$, stop.
\item or \textbf{Bayes factor}: If Bayes factor is higher than 3 or smaller than 1/3, stop.
\item or \textbf{Bayes precision}: If width of credible interval is smaller than 0.08, stop. 
\end{itemize}


\section{Evaluation}

\subsection{Evaluate Significance Analysis}
We are going to compare FP, TP, FN, TN for the three significance analysis algorithms described in section ~\ref{sec:byt}. 

\subsection{Evaluate Early Stopping Criteria}
It is obvious that if we use the frequentist approach in early stopping, we should also use confidence interval to make conclusion of significance. However, things get a bit more complicated in the Bayesian case. When using Bayes factor to stop, we often find a conflicting result if we draw significance based on credible interval. Find below a table of combination of early stopping algorithms and significance analysis algorithms we will evaluate on.

\begin{center}
  \begin{tabular}{ | r | c | c | c | c | }
    \hline
    \multirow{2}{*}{Significant Based On} & \multicolumn{4}{ |c| }{Early Stopping Based On} \\ \cline{2-5}
    & Confidence Interval & Credible Interval & Bayes factor & Bayes precision \\ \hline
    Confidence Interval & \Checkmark & &  &  \\ \hline
    Credible Interval &  & \Checkmark &  &  \\ \hline
    Bayes factor &  & & \Checkmark & \Checkmark \\ \hline
  \end{tabular}
\end{center}

For each combination shown in the table above, we will evaluate the following metrics:
\begin{itemize}  
\item \textbf{False positive rate}: Percentage of wrongly stopped experiment. i.e., Early stop says there is a significant difference and stops the experiment, but a regular test will tell you not significant when the sample size is reached.
\item \textbf{Run time reduced}: Average run time reduced for the correctly stopped experiment.
\item \textbf{Bias}: Percentage of the difference of effect size (delta/std).
\item \textbf{False positive run time reduced}: Average run time reduced for the wrongly stopped experiment.
\end{itemize}

\section{Result}

\subsection{Simulation Data}
We generate 2000 simulation tests based on Gaussian distributed KPIs. We calculate the minimal detectable effect size calculated from power analysis. 1000 tests should have a significant difference between control and treatment, and the other 1000 tests should have no significant difference. 

We simulate the A/B testing for a period of 20 days, where the frequency of the visit from an entity is modelled by a Poisson distribution with visits on 3 days per entity in average. We run an analysis and evaluate whether to stop on each day.

\subsection{Real Data}
Use real data from previous A/B testings.

\section{Conclusion}
Use group sequential and confidence interval.

%----------------------------------------------------------------------------------------

\end{document}