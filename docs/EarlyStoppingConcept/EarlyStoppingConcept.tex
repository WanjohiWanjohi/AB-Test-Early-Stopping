%%%%%%%%%%%%%%%%%%%%%%%%%%%%%%%%%%%%%%%%%
% Short Sectioned Assignment
% LaTeX Template
% Version 1.0 (5/5/12)
%
% This template has been downloaded from:
% http://www.LaTeXTemplates.com
%
% Original author:
% Frits Wenneker (http://www.howtotex.com)
%
% License:
% CC BY-NC-SA 3.0 (http://creativecommons.org/licenses/by-nc-sa/3.0/)
%
%%%%%%%%%%%%%%%%%%%%%%%%%%%%%%%%%%%%%%%%%

%----------------------------------------------------------------------------------------
%	PACKAGES AND OTHER DOCUMENT CONFIGURATIONS
%----------------------------------------------------------------------------------------

\documentclass[paper=a4, fontsize=11pt]{scrartcl} % A4 paper and 11pt font size

\usepackage[T1]{fontenc} % Use 8-bit encoding that has 256 glyphs
\usepackage{fourier} % Use the Adobe Utopia font for the document - comment this line to return to the LaTeX default
\usepackage[english]{babel} % English language/hyphenation
\usepackage{amsmath,amsfonts,amsthm} % Math packages

\usepackage{lipsum} % Used for inserting dummy 'Lorem ipsum' text into the template

\usepackage{sectsty} % Allows customizing section commands
\allsectionsfont{\centering \normalfont\scshape} % Make all sections centered, the default font and small caps

\usepackage{fancyhdr} % Custom headers and footers
\pagestyle{fancyplain} % Makes all pages in the document conform to the custom headers and footers
\fancyhead{} % No page header - if you want one, create it in the same way as the footers below
\fancyfoot[L]{} % Empty left footer
\fancyfoot[C]{} % Empty center footer
\fancyfoot[R]{\thepage} % Page numbering for right footer
\renewcommand{\headrulewidth}{0pt} % Remove header underlines
\renewcommand{\footrulewidth}{0pt} % Remove footer underlines
\setlength{\headheight}{13.6pt} % Customize the height of the header

\numberwithin{equation}{section} % Number equations within sections (i.e. 1.1, 1.2, 2.1, 2.2 instead of 1, 2, 3, 4)
\numberwithin{figure}{section} % Number figures within sections (i.e. 1.1, 1.2, 2.1, 2.2 instead of 1, 2, 3, 4)
\numberwithin{table}{section} % Number tables within sections (i.e. 1.1, 1.2, 2.1, 2.2 instead of 1, 2, 3, 4)

\setlength\parindent{0pt} % Removes all indentation from paragraphs - comment this line for an assignment with lots of text

\usepackage{amsmath}

%----------------------------------------------------------------------------------------
%	TITLE SECTION
%----------------------------------------------------------------------------------------

\newcommand{\horrule}[1]{\rule{\linewidth}{#1}} % Create horizontal rule command with 1 argument of height

\title{	
\normalfont \normalsize 
\textsc{} \\ [25pt] % Your university, school and/or department name(s)
\horrule{0.5pt} \\[0.4cm] % Thin top horizontal rule
\huge Bayesian Hypothesis Testing in ExpAn  \\ % The assignment title
\horrule{2pt} \\[0.5cm] % Thick bottom horizontal rule
}

\author{Shan Huang} % Your name

\date{\normalsize\today} % Today's date or a custom date

\begin{document}

\maketitle % Print the title

%----------------------------------------------------------------------------------------
%	PROBLEM 1
%----------------------------------------------------------------------------------------

This documentation explains how Bayes factor is implemented in ExpAn. 

This is not a formal paper. The intention is to share knowledge between developers internally. 

\section{Problem}
Given samples $\textbf{x}$ from treatment group, samples $\textbf{y}$ from control group, we want to know whether there is a statistically significant difference between the mean of two variants.

%We are interested in the difference of treatment and control: $z = y-x$.

\section{ Model setup}

We assume that
\begin{align} 
\begin{split}
x &\sim \mathcal{N}(\mu + \alpha, \sigma^2)\\
y &\sim \mathcal{N}(\mu, \sigma^2)
\end{split}					
\end{align}

$\delta$ represents how many units of std is the mean difference:
\begin{align}
\begin{split}
\delta  &= \frac{\alpha}{ \sigma}
\end{split}					
\end{align}
 
We place non-informative priors on the data model:
\begin{align}
\begin{split}
\mu &\sim \mathit{Cauchy} (0, 1) \\
\sigma &\sim \mathit{Gamma} (2, 2)
\end{split}
\end{align}

\section{Bayesian hypothesis testing}
\label{sec:byt}
We are interested in the "standardised" difference $\delta$. Therefore, we need to make hypothesis on $\delta$ and compare the hypothesis. 

Our null hypothesis of $\delta$ is always a fixed value $0$, which means we believe there is no difference at all. On the other hand, our alternative hypothesis $H_1$ is a probability distribution, such that we can update the posterior $p(H_1 |D)$, or $p_1(D)$, after seeing the data. The prior distribution of $\delta$ is a vague cauchy.
\begin{align}
\begin{split}
H_0:  \delta &= 0 \\
H_1:  \delta &\sim \mathit{Cauchy} (0, 1)
\end{split}	
\end{align}

After seeing the data, we update the posterior distribution $p(H_1 |D)$ represented by $p_1( \delta | D)$. 
We compute the 95\% highest density interval of the posterior $p_1( \delta | D)$ as credible interval of $\delta$. We can then use the credible interval of $\delta$ to compute credible interval of $\alpha$. Finally, credible interval of $\alpha$ can be used to decide whether the result is statistically significant (whether 0 is outside the interval).


\section{Early stopping}

\subsection{Stop by Bayes precision}
We can use the width of credible interval as the stop criteria, because the smaller the interval is, the more certain we are about the posterior.
If we are certain about our posterior, then we can stop. ;)

From best practice, we set the threshold of width to 0.08.

\subsection{Stop by Bayes factor}
Bayes factor simply compares the ratio of likelihood:
\begin{align}
\begin{split}
BF_{01} = \frac{p(D|H_0)}{p(D|H_1)}
\end{split}
\end{align}

$BF_{01}$ smaller than 1/3 can be interpreted as support for the null hypothesis (no difference),  higher than 3 can be interpreted as support for the alternative hypothesis (significant difference).  Values between 1/3 and 3 are inconclusive. 

The threshold $BF_{01}  >3$ or $BF_{01}  < \frac{1}{3}$ are heuristic values from best practice. The definition of strong evidence is subjective, one can also use a threshold of 10 or more.

\subsection{Stop by credible interval}
Note that we can also stop and/or analyse siginificance by credible interval as described in section~\ref{sec:byt}. A comparison of different stopping criteria and significance analysis will be summarised in another documentation.

(Not implemented yet.)

\section{Savage-Dickey density ratio}
Since the likelihood in Bayes factor involves an intractable integral over $\delta$, we use Savage-Dickey density ratio to compute Bayes factor in implementation. 

For simplicity, we use subscripts 0 and 1 to denote the probability densities under hypothesis $H_0$ and $H_1$, respectively:
\begin{align}
\begin{split}
BF_{01} = \frac{p_0(D)}{p_1(D)} = \frac{p_1(\delta=0 | D)}{p_1(\delta=0)}
\end{split}
\end{align}

\emph{Proof}:

Assume the model parameter falls into two categories $\theta = (\phi, \psi)$.

$\phi$ are the parameters of interest for the hypothesis testing. In our model setup, our $\phi$ is $\delta$. We further assume under null hypothesis $\phi$ is set to a special fixed value $\phi = \phi_0$, whereas the alternative hypothesis places a distribution over $\phi$ --- the same setup in section~\ref{sec:byt}.

$\psi$ are so-called nuisance parameters. We don't care about the value $\psi$ in the experiment. We introduce $\psi$ here only for the reason of deriving the formula of Savage-Dickey density ratio. \\

Now assume that $p( \psi | \phi)$ does not depend on the model. Then if $\phi$ is continuous at $\phi_0$, 
\begin{align}
\begin{split}
\lim_{\phi \rightarrow \phi_0} {p_1( \psi | \phi) = p_0(\psi | \phi_0)} 
\end{split}
\end{align}

By setting $\phi = \phi_0$ for $H_1$, and $\phi = \phi_0$ by definition for $H_0$, we get a lemma:
\begin{align}
\begin{split}
p_1( \psi | \phi = \phi_0) = p_0(\psi)
\end{split}
\end{align}

The marginal likelihood over $\psi$ under $H_0$ is given by
\begin{align}
\begin{split}
p_0(D) = \int \! p_0(D | \psi) p_0(\psi) \, \mathrm{d}\psi.
\end{split}
\end{align}

Using the continuity condition and our lemma, this can be rewritten as
\begin{align}
\begin{split}
p_0(D) = \int \! p_1(D | \psi, \phi=\phi_0) p_1(\psi | \phi = \phi_0) \, \mathrm{d}\psi.
\end{split}
\end{align}

Note that the right-hand side of the equation above is a marginalization over $\psi$, thus $\psi$ disappears!
\begin{align}
\begin{split}
p_0(D) = p_1(D | \phi = \phi_0)
\end{split}
\end{align}

Finally, we apply Bayes' rule to get
 \begin{align}
\begin{split}
p_0(D) = \frac{p_1(\phi = \phi_0 | D) p_1(D)}{p_1(\phi = \phi_0)} 
\end{split}
\end{align}

Substituting $p_0(D)$ in the equation of Bayes factor
\begin{align}
\begin{split}
BF_{01} = \frac{p(D|H_0)}{p(D|H_1)} =\frac{p_0(D)}{p_1(D)} = \frac{p_1(\phi = \phi_0 | D)}{p_1(\phi = \phi_0)}
\end{split}
\end{align}

$\blacksquare$

%----------------------------------------------------------------------------------------

\end{document}