%%%%%%%%%%%%%%%%%%%%%%%%%%%%%%%%%%%%%%%%%
% Short Sectioned Assignment
% LaTeX Template
% Version 1.0 (5/5/12)
%
% This template has been downloaded from:
% http://www.LaTeXTemplates.com
%
% Original author:
% Frits Wenneker (http://www.howtotex.com)
%
% License:
% CC BY-NC-SA 3.0 (http://creativecommons.org/licenses/by-nc-sa/3.0/)
%
%%%%%%%%%%%%%%%%%%%%%%%%%%%%%%%%%%%%%%%%%

%----------------------------------------------------------------------------------------
%	PACKAGES AND OTHER DOCUMENT CONFIGURATIONS
%----------------------------------------------------------------------------------------

\documentclass[paper=a4, fontsize=11pt]{scrartcl} % A4 paper and 11pt font size

\usepackage[T1]{fontenc} % Use 8-bit encoding that has 256 glyphs
\usepackage{fourier} % Use the Adobe Utopia font for the document - comment this line to return to the LaTeX default
\usepackage[english]{babel} % English language/hyphenation
\usepackage{amsmath,amsfonts,amsthm} % Math packages

\usepackage{lipsum} % Used for inserting dummy 'Lorem ipsum' text into the template

\usepackage{sectsty} % Allows customizing section commands
\allsectionsfont{\centering \normalfont\scshape} % Make all sections centered, the default font and small caps

\usepackage{fancyhdr} % Custom headers and footers
\pagestyle{fancyplain} % Makes all pages in the document conform to the custom headers and footers
\fancyhead{} % No page header - if you want one, create it in the same way as the footers below
\fancyfoot[L]{} % Empty left footer
\fancyfoot[C]{} % Empty center footer
\fancyfoot[R]{\thepage} % Page numbering for right footer
\renewcommand{\headrulewidth}{0pt} % Remove header underlines
\renewcommand{\footrulewidth}{0pt} % Remove footer underlines
\setlength{\headheight}{13.6pt} % Customize the height of the header

\numberwithin{equation}{section} % Number equations within sections (i.e. 1.1, 1.2, 2.1, 2.2 instead of 1, 2, 3, 4)
\numberwithin{figure}{section} % Number figures within sections (i.e. 1.1, 1.2, 2.1, 2.2 instead of 1, 2, 3, 4)
\numberwithin{table}{section} % Number tables within sections (i.e. 1.1, 1.2, 2.1, 2.2 instead of 1, 2, 3, 4)

\setlength\parindent{0pt} % Removes all indentation from paragraphs - comment this line for an assignment with lots of text

\usepackage{amsmath}

%----------------------------------------------------------------------------------------
%	TITLE SECTION
%----------------------------------------------------------------------------------------

\newcommand{\horrule}[1]{\rule{\linewidth}{#1}} % Create horizontal rule command with 1 argument of height

\title{	
\normalfont \normalsize 
\textsc{} \\ [25pt] % Your university, school and/or department name(s)
\horrule{0.5pt} \\[0.4cm] % Thin top horizontal rule
\huge Bayesian Hypothesis Testing in ExpAn  \\ % The assignment title
\horrule{2pt} \\[0.5cm] % Thick bottom horizontal rule
}

\author{Shan Huang} % Your name

\date{\normalsize\today} % Today's date or a custom date

\begin{document}

\maketitle % Print the title

%----------------------------------------------------------------------------------------
%	PROBLEM 1
%----------------------------------------------------------------------------------------

This documentation explains how Bayes factor is implemented in ExpAn. 

It is not a formal paper. The intention of the documentation is to share knowledge between developers internally, and I want to keep it short. 

\section{Problem}
Given samples $\textbf{x}$ from treatment group, samples $\textbf{y}$ from control group\footnote{Control is the group for baseline metric. So $y$ is the baseline. We add a random variable on $y$ to get $x$. This is somewhat confusing to me, but it is the terminology we are using throughout ExpAn. }, we want to know whether there is a statistically significant difference.

%We are interested in the difference of treatment and control: $z = y-x$.

\section{ Model setup}

We assume that
\begin{align} 
\begin{split}
x &\sim \mathcal{N}(\mu + \alpha, \sigma^2)\\
y &\sim \mathcal{N}(\mu, \sigma^2)
\end{split}					
\end{align}

The "z-scored" or "standardised" difference we are interested in is:
\begin{align}
\begin{split}
\delta  &= \frac{\alpha}{ \sigma}
\end{split}					
\end{align}
 
Before seeing any data, we place the following non-informative priors on the parameters:
\begin{align}
\begin{split}
\mu &\sim \mathit{Cauchy} (0, 1) \\
\sigma &\sim \mathit{Gamma} (2, 2)
\end{split}
\end{align}

\section{Bayesian hypothesis testing}
\label{sec:byt}
We are interested in the "standardised" difference $\delta$. Therefore, we need to make hypothesis on $\delta$ and compare the hypothesis. 

Note that our null hypothesis of $\delta$ is always a fixed value $0$, which means we believe there is no difference at all. On the other hand, our alternative hypothesis $H_1$ is a probability distribution, such that we can update the posterior $p(H_1 |D)$, or in other notation $p_1(D)$, after seeing the data.
\begin{align}
\begin{split}
H_0:  \delta &= 0 \\
H_1:  \delta &\sim \mathit{Cauchy} (0, 1)
\end{split}	
\end{align}

After seeing the data, we compute the posterior distribution $p(H_1 |D)$ represented by $p_1( \delta | D)$. This is done via MCMC sampling from \texttt{StanModel.sampling}, which is pretty slow\footnote{Is there a better way? Maybe there exist an analytical solution.} and might have some potential improvements.

After getting the posterior, we compute the .95 posterior density interval and show it in our result object in column \texttt{pctile} and \texttt{value}. This interval is computed via another MCMC sampling, here we implemented this method ourselves in \texttt{early\_stopping.HDI\_from\_MCMC}.



\section{Early stopping}

\subsection{Bayes precision}
We use the width of .95 posterior interval as the stop criteria. The smaller the density interval is, the more certain we are about the posterior.
If we are certain about our posterior, then we can stop. ;)

For some reason, we set the threshold of width to 0.08.

\subsection{Bayes factor}
Bayes factor simply compares the ratio of likelihood:
\begin{align}
\begin{split}
BF_{01} = \frac{p(D|H_0)}{p(D|H_1)}
\end{split}
\end{align}

If this value is big or small enough (which means the difference between two hypothesis is pretty large), we make the decision of early stopping.
For some reason, we set the threshold to a heuristic value: $BF_{01}  >3$ or $BF_{01}  < \frac{1}{3}$.

Note that our analysis result is always the .95 credible interval as described in section~\ref{sec:byt}. We only use Bayes factor for the decision of early stopping. 

\subsection{Savage-Dickey density ratio}
Since the likelihood in Bayes factor involves an intractable integral over $\delta$, we use Savage-Dickey density ratio to compute Bayes factor in implementation. 

For simplicity, we use subscripts 0 and 1 to denote the probability densities under hypothesis $H_0$ and $H_1$, respectively:
\begin{align}
\begin{split}
BF_{01} = \frac{p_0(D)}{p_1(D)} = \frac{p_1(\delta=0 | D)}{p_1(\delta=0)}
\end{split}
\end{align}

\emph{Proof}:

Assume the model parameter falls into two categories $\theta = (\phi, \psi)$.

$\phi$ are the parameters of interest for the hypothesis testing. In our model setup, our $\phi$ is $\delta$. We further assume under null hypothesis $\phi$ is set to a special fixed value $\phi = \phi_0$, whereas the alternative hypothesis places a distribution over $\phi$ --- the same setup in section~\ref{sec:byt}.

$\psi$ are so-called nuisance parameters. We don't care about the value $\psi$ in the experiment. We introduce $\psi$ here only for the reason of deriving the formula of Savage-Dickey density ratio. \\

Now assume that $p( \psi | \phi)$ does not depend on the model. Then if $\phi$ is continuous at $\phi_0$, 
\begin{align}
\begin{split}
\lim_{\phi \rightarrow \phi_0} {p_1( \psi | \phi) = p_0(\psi | \phi_0)} 
\end{split}
\end{align}

By setting $\phi = \phi_0$ for $H_1$, and $\phi = \phi_0$ by definition for $H_0$, we get a lemma:
\begin{align}
\begin{split}
p_1( \psi | \phi = \phi_0) = p_0(\psi)
\end{split}
\end{align}

The marginal likelihood over $\psi$ under $H_0$ is given by
\begin{align}
\begin{split}
p_0(D) = \int \! p_0(D | \psi) p_0(\psi) \, \mathrm{d}\psi.
\end{split}
\end{align}

Using the continuity condition and our lemma, this can be rewritten as
\begin{align}
\begin{split}
p_0(D) = \int \! p_1(D | \psi, \phi=\phi_0) p_1(\psi | \phi = \phi_0) \, \mathrm{d}\psi.
\end{split}
\end{align}

Note that the right-hand side of the equation above is a marginalization over $\psi$, thus $\psi$ disappears!
\begin{align}
\begin{split}
p_0(D) = p_1(D | \phi = \phi_0)
\end{split}
\end{align}

Finally, we apply Bayes' rule to get
 \begin{align}
\begin{split}
p_0(D) = \frac{p_1(\phi = \phi_0 | D) p_1(D)}{p_1(\phi = \phi_0)} 
\end{split}
\end{align}

Substituting $p_0(D)$ in the equation of Bayes factor
\begin{align}
\begin{split}
BF_{01} = \frac{p(D|H_0)}{p(D|H_1)} =\frac{p_0(D)}{p_1(D)} = \frac{p_1(\phi = \phi_0 | D)}{p_1(\phi = \phi_0)}
\end{split}
\end{align}

$\blacksquare$

%----------------------------------------------------------------------------------------

\end{document}